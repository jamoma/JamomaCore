% -----------------------------------------------
% Template for SMC 2009
%     smc2009.sty -> style file
% Last modified by Fabien Gouyon (smc2009@inescporto.pt)
% Modified by Juan P. Bello (ismir2008-papers@ismir.net)
% By Rainer Typke (ismir07.rainer@safersignup.com)
% Based on the 2004 template by Eloi Batlle.
% -----------------------------------------------

\documentclass{article}
\usepackage{smc2009,amsmath}
% To use when using pdflatex
\usepackage{graphicx}
\usepackage{url}     
\usepackage{hyperref}  
      
\newenvironment{packed_item}{
\begin{itemize}
  \setlength{\itemsep}{1pt}
  \setlength{\parskip}{0pt}
  \setlength{\parsep}{0pt}
}{\end{itemize}}

\newenvironment{packed_enumerate}{
\begin{enumerate}
  \setlength{\itemsep}{1pt}
  \setlength{\parskip}{0pt}
  \setlength{\parsep}{0pt}
}{\end{enumerate}}
% To use when using latex, dvips and ps2pdf
% \usepackage[dvips]{graphicx}

% Title.
% ------
%\title{a layered approach to sound spatialization - concepts and examples}
\title{A stratified approach for sound spatialization}
% IMPORTANT NOTICE:
% Reviews are double-blind
% Authors will not be informed of who reviews their papers, and author names will be concealed from the reviewers 
% Please avoid evident self references in the text

% Authors' names must be omitted from title page (or listed as �name(s) omitted for submission�)


% Single address
% To use with only one author or several with the same address
% ---------------
%\oneauthor
%   {Author} {School \\ Email}

% Two addresses
 %--------------
\twoauthors
  {First author} {School \\ Email}
  {Second author} {Company \\ Email}

% Three addresses
% --------------
%\threeauthors
%  {First author} {School \\ Email}
%  {Second author} {Company \\ Email}
%  {Third author} {Company \\ Email}


\begin{document}
%    
\sloppy
\maketitle
%

\permission

\begin{abstract}  
We propose a multi-layer approach to mediate across essential components involved in sound spatialization. This approach will facilitate artistic work with spatialization systems, a process which currently lacks structure, flexibility, and interoperability. 
\\                            	
%\textbf{Keywords:} sound spatialization, interoperability, layered architecture, adaptation, site-specific, Ambisonics , TODO: make better \& stronger keywords   
%-flexible working environment for spatialization\\
%-A layered approach towards interoperability in sound spatialization management\\
%The abstract should be placed at the top left column and should contain
%about 150-200 words.
%stratified     strata  stratum 
%KEYWORDS : 
\end{abstract}

\section{Introduction}\label{sec:introduction}         
The improvements in computer and audio equipment in recent years make it possible to experiment more freely with resource-demanding sound synthesis techniques such as spatial sound synthesis, also known as spatialization. For seeking new means of expression, different spatialization applications should be readily combined and accessible for both programmatic and user interfaces. % and/or other controllers. %e.g. a generative mapping of a granular synth output to different spatial renderer from one common user interface, while also streaming spatial encoded audio to the internet. 
Furthermore, 
%The goal of any spatialization
%rendering is it only the rendering system, or control/rendering ?
%system is to facilitate or empower the creative use of the spatial medium for a sonic artist. 
quantitative studies on spatial music (\cite{otondo2008ctu}, \cite{PetersSurvey}) remind us that there are great individual and context-related differences in the compositional use of spatialization and that there is no one spatialization system that could satisfy every artist. %that the compositional use of spatialization techniques varies strongly across composers and musical context.    
For instance, the requirements of a computer aided spatialization system may vary between a fixed-media composition (e.g \cite{BarrettOS02}%TODO: find a Robert Normandeau reference
), an art-installation (\cite{lossius:2007sound_space_body}), and a live diffusion performance (\cite{Truax99}). 
One example is an interactive art installation where the real-time quality of a spatial rendering system is of great importance, in combination with the possibility to control spatial processes through a multi-touch screen.  Juxtaposed is a second example: a performance of a fixed-media composition where the paramount features may be multichannel playback and the compensation of non-equidistant loudspeakers (in terms of sound pressure and time delays). Additional scenarios may require binaural rendering for headphone listening, multichannel recording, up and down mixing, or a visual representation of a sound scene.  
Moreover, even during the creation of one spatial art work, the importance of these requirements may change throughout different stages of the creative processes. % \emph{Experimentation -- Arranging -- Performance -- Documentation}. 
%Several common paradigms for spatial sound synthesis are outlined in section \ref{sec:review}. %The availability of better CPU resources in recent years makes it possible to experiment more freely with these paradigms for seeking new ways of expressions, e.g. a generative mapping of a granular synth output to different spatial renderer from one common user interface, while also streaming spatial encoded audio to the internet. This paper shows how such complex spatialization scenarios can be conceptualized through a multi-layer structure.    


Guaranteeing efficient workflow for sound spatialization requires structure, flexibility, and interoperability across all involved components. As outlined in Section \ref{sec:review}, common spatialization systems are too often self-contained, giving no consideration to these requirements. Therefore, we propose a multi-layer approach to mediate sound spatialization to meet these demands. 


%\begin{packed_item} 
%	\item {real-time or non real-time rendering}  
%%	\item {Plug-In structure: exchangeable renderer and interface components} - this is one of the ideas of the article, we don't need it here
%	\item {multichannel playback and recording possibility} 	
%	\item {binaural rendering for headphone listening}
%	\item {testing technical setup, e.g. loudspeaker connections}   
%	\item {customizable renderer in order to accommodate for different acoustical conditions, e.g. adapting the virtual room description to the listening room} 
%\item {customizable to accommodate for different technical conditions, e.g. rendering to different reproduction formats, compensation for non-ideal loudspeaker configurations, routing signals to dedicated physical outputs}  
%	\item {visual representation of a sound scene} 
%\item {separate render and control layer} - this is one of the ideas of the article, we don't need it here
%\item {allowing for external control e.g. through MIDI, OSC or other protocols}
%
%\end{packed_item}
%TODO: this list must become better!-please help - \emph{what do we want to say with this list ??} \\

%The outlined requirements may differ according to artistic paradigms.
%-Use cases (users : composers, installation artists, scientists...etc...)
%   - OpenMusic off-line rendering
%   - BEAST sound diffusion (the more speaker - the mre complicated to perform)
%   - interactive sound installations
%   - real-time control via sensors/gestural controller etc. by performer
%   - tape-music (fixed media)
%   - computer generated - real time spatialization (see <meta-description> layer)
   
    
%Experimental stage:\\
%\begin{packed_item}
% \item {Plug-In structure: exchangeable renderer and interface components} 
% \item {multichannel recording possibility for capturing of sketches/ideas}         
% \item {present management}
% \item {sound scene visualization - especially for off-line rendering of spatial processes}
%\end{packed_item} 

%Compositional stage:\\  
%\begin{packed_item}
%  \item {binaural rendering for headphone listening} 
%\end{packed_item} 
%Performance stage:\\
%\begin{packed_item} 
%	\item {customizable to accommodate for different technical conditions, e.g. rendering to different reproduction formats, compensation for non-ideal %loudspeaker configurations, routing signals to dedicated physical outputs}   
%    \item {testing technical setup, e.g. loudspeaker connections}
%    \item {customizable to accommodate for different acoustical conditions, e.g. adapting the virtual room description to the listening room} 
%\end{packed_item} 
%Documentation stage:\\  
%\begin{packed_item}
%\item {multichannel recording possibility}         
%\item {storing }       
%\end{packed_item}  


%\quote{ What WE need, for our personal work, is a way to extend the capabilities of those tools in a completely flexible and configurable way - and that suggests plug-ins (though it will always be a potential problem overcoming inherent IO structures in the host applications).}

%\quote{working with non-standard loudspeakers: Meyer Sound Spherical Loudspeaker Array that is under research at CNMAT, or the Hemisphere Point-source Emanation Loudspeaker, for example.  I would like to experiment with these;}

%\quote{Tool building and music making happen together and depend upon each other.}

%\quote{Very frustrated with my current spatialization software and am desperately looking for something better!}

\section{Review of current Paradigms} \label{sec:review}
%This section gives an overview of current paradigms and solutions for sound spatialization, and shows their limitations. 
%\subsection{Live sound diffusion}  
%?? shall we mention live diffusion as a paradigm
%[PB]: Sounds, sensible... I  could write a little something about electroacoustic music live diffusion, as well as multi-channel sound design practices... and I can ask Benjamin for a sentence or two about live spatialization, e.g. of acoustic ensembles.
%Does that sound relevant ? Of course, all of these are kind of "legacy techniques", mainly analog and depending on the mixing console... but there may be some parallels... Any Opinion ?

\subsection{Digital Audio Workstations - DAW} \label{sec:digital audio workstations - DAW} 
Currently, many composers and sound designers use DAWs for designing their sound spatialization primarily in the context of fixed media, tape-music, and consumer media production. Users of DAWs often have former experience with live diffusion systems, or with panning in a hardware mixing console. Their migration to DAWs seems reasonable because the linear time representation of sound material and the audio bus architecture in DAWs originates from the use of multitrack tape recorder and mixing consoles. 


A number of (mainly commercial) DAWs are mature and offer a systematic user interface, good project and sound file management, and extendability through plug-ins so that they can fulfill the needs of many users in the described context.\\
However, through focusing on consumer media products, multichannel capabilities are limited. ITU 5.1 \cite{ITU:1993_surround_5:1}, a surround sound format with equidistant loudspeakers around an ideal located listener is the most common multichannel format. % in DAWs and in surround panning plugins. 
Its artistic use may be limited because 5.1 favors the frontal direction and has reduced capabilities for localizing virtual sources from the sides and back.  More recently, extensions up to 10.2 are available\footnote{A comparison of DAWs concerning their multichannel audio capabilities shows \scriptsize{\url{http://acousmodules.free.fr/hosts.htm}}.}.   
Also, in art installations or concert hall environments, non-standard loudspeaker setups are common due to artistic or practical reasons, varying in number and arrangements of loudspeakers.  These configurations are typically unaccounted in DAWs and therefore often difficult to use. 
%From 1993, emergence of the ITU 5.1 has been announced as a major improvement for mixing surround sound. Originally designed to fulfill the needs of the film industry, this loudspeaker configuration appeared to music and entertainment companies as a new way to extend their commercial offer. Since then, all DAWs surround panners have been designed to respect and organize sound processing for this set-up. It was not until the late nineties that new suggestions for surround workstations (from 6.1 to 10.2, enhancing sides precision, and perception outside of the Central Listening Position) were considered.\\
%This was an useful approach for recording, multimedia or film prospects (cannot do business without standards), but still in a regular, circular, geometric and (FIXME: non-neutral ?we mean it influences the user) design, which cannot be acceptable and sufficient for creation or live applications. As an example, a non-flexible loudspeaker circle set-up is very rarely adapted to the concert hall.\\ 


DAW surround panners often comprise a parameter named \emph{blur}, \emph{divergence}, or \emph{spread} that controls the apparent source width through modifying the distributed the sound energy among loudspeakers. Although this parameter enriches the creative possibilities, it is not always integrated in DAWs. %and because we would work with this parameter considering the specific features of the sound we have to spatialize. 
Moreover, this parameter is often only indirectly accessible, e.g. through changing the distance of the sound source (Apple Logic Pro).%, and is related to parameters such as \emph{spread} or \emph{diversity} (e.g. in Apple's Logic Pro). To the authors' opinion, the most flexible control element can be found in Sequoia \footnote{\url{http://www.magix.com/us/sequoia/}}, with \emph{soundfield} offset and character (shape) settings for each track, a \emph{global divergence} to fit to the listening room size, and a complete flexibility for the speaker set-up in the listening area. 
%
%
% Did we discuss all point below:? (moved above -> we should use them at the beginning IMHO)
%
%
%spatialization is mainly done in DAWs and leads to "panning" because the bus architectures and interfaces of the panning plugins suggests this.
%    Limitations:
%        a) mainly tied to consumer formats (stereo and ITU 5:1 surround)
%        b) restricted to linear prerendering compositional processes
%        c) complicated to maintain automation when changing rendering plugIn
%        d) burden to connect interfaces (Lemur, Stantum, Wacom, Camera-tracking)
%        
% 
      
\subsection{Media programming environments}  \label{sec:Media programming environments}

Beside the paradigm of the DAW, various media programming environments exist that are capable of spatial sound synthesis. These include SuperCollider, Pure Data, Chuck, OpenMusic, and Max/MSP.  In order to support individual approaches and to meet the specific needs of computer music and mixed media art, these environments enable the user to combine music making with computer programming.

 
However, in the name of complete flexibility, these environments lack in providing structured solutions for the specific challenges of spatial music as outlined in section \ref{sec:introduction}. Consequently, numerous self-contained spatialization libraries and toolboxes have been created by artists and researchers to generate virtual sound sources and artificial spaces, such as Space Unit Generator \cite{SUG02}, Spatialisateur \cite{JotPhD}, or ViMiC \cite{CMJ08-VIMIC}. Also toolboxes dedicated to sound diffusion practice, such as the BEASTmulch System\footnote{\url{http://www.beast.bham.ac.uk/research}}, or ICAST \cite{ICAST06} has been developed. Each tool, however, may only provide solutions for a subset of compositional viewpoints. The development of new aesthetics through combining these tools is difficult or limited through their specific designs. 

\subsection{Stand-alone Applications}
A variety of powerful stand-alone spatialization systems are in development, ranging from directional based spatialization frameworks (e.g. SSR \cite{geier2008ssr}, Zirkonium \cite{ramakrishnan2006zk}) and Auditory Virtual Environments (AVE, e.g. tinyAVE \cite{BorssAES35}) to sound diffusion and particle oriented approaches (e.g. Scatter \cite{ScatterSMC08}). Although these applications usually promote their embedded graphical user interfaces as the primary method to access their embedded DSP-algorithms, a few strategies to allow communication from outside through self-contained  XML, MIDI or OSC protocols can be found.  


%%%%%%%%%%%%%%%%%%%%%%%%%%%%%%%%%%%%%%%%%%%%%%%%%%%%%%%%%%%%%%%%%%%%%%%%%%%%%
%
%%%%%%%%%%%%%%%%%%%%%%%%%%%%%%%%%%%%%%%%%%%%%%%%%%%%%%%%%%%%%%%%%%%%%%%%%%%%%

\section{A stratified approach to the spatialization workflow}  
 
When dealing with spatialization in electroacoustic composition or linear sound editing, the workflow comprises a number of steps in order to construct, shape and realize the spatial qualities of the work. The creative workflow might appear to be different when working on audio installations or interactive/multimedia work. Still we identified underlying common elements that are always in play when spatialization is used. For this reason a layered approach is proposed, where the required processes are organized according to levels of abstraction.

This model is inspired by the OSI network model\footnote{\url{http://en.wikipedia.org/wiki/OSI_model}}, which is an abstract description for layered communications and computer network protocol design. OSI (Open Systems Interconnection) divides network architecture into seven layers that range from top to bottom between the Application and Physical Layers. Each OSI-layer contains a collection of conceptually similar functionalities that provide services to the layer above it and receives service from the layer below it. \\
\indent In our proposed model, depicted in Figure \ref{fig:layers}, six layers have been identified. The adaptation of concepts originally designed for network protocols to computer music systems was legitimized, for instance, in creating the popular Open Sound Control (OSC) protocol \cite{wright1997osc}.   

	\begin{figure}
		[ht] \centerline{\framebox{ 
		\includegraphics[width=0.93\columnwidth]{layers2.pdf}}} \caption{Layers and streams in sound spatialization} \label{fig:layers} 
	\end{figure}

\subsection{Physical Device Layer}

The major functionality of this layer is to establish the acoustical connection between computer and listener. 
                                                     
It defines the electrical and physical specifications of devices that creates the acoustical signals, such as soundcards, amplifiers, loudspeakers, and headphones.

\subsection{Hardware Abstraction Layer}

This layer contains the audio services that runs in the background of a computer OS and manages multichannel audio data between the physical devices and higher layers.

Core Audio, ALSA, or PortAudio are examples of such services. Extensions such as JACK, Soundflower, Rewire and networked audio streaming can be used for more complex distributions of audio signals among different audio clients.

\subsection{Encoding and Decoding Layers}	

In the proposed model the spatial rendering is considered to consist of two layers. The \emph{Encoding Layer} produce encoded signals containing spatial information while remaining independent of and unaware of the speaker layout. The decoding process interprets the encoded signal and decodes it for the speaker layout at hand. Wiggins \cite[p.~99]{Wiggins2004PhDThesis} argues for the benefits of such hierarchical rendering methods:

\begin{packed_item} 
	\item {The created piece will be much more portable in that, as long as a decoder is available, many different speaker layouts can be used.}  
	\item {The recordings will become more future-proof as, if a speaker layout changes, just a re-decode is needed, rather than a whole remix of the piece.} 	
	\item {The composition/recording/monitoring of the piece will become more flexible as headphones, or just a few speakers can be used. This will result in less space being needed. This is particularly useful for on-location recordings, or small studios, where space may be limited.}
\end{packed_item}

Examples of such hierarchical rendering methods are Ambisonics B-Format, Higher Order Ambisonics, DIRAC \cite{Pulkki2007dirac_1}, MPEG Surround, AC-3, or DTS.

Not every rendering technique generate intermediate encoded signals, but instead can be considered to encapsulate the \emph{Encoding} and \emph{Decoding Layers} in one process. Some examples of such renderers are VBAP \cite{Pulkki:1997vbap}, DBAP \cite{dbapICMC09}, ViMiC \cite{CMJ08-VIMIC} and Ambisonics equivalent panning \cite{Neukom:2008ambipan}.

Processing of sources to create an impression of distance, such as Doppler effect, gain attenuation and air absorption filters, are considered to belong to the encoding layer. Likewise early reflections and reverberation belongs in the encoding layer. The Waves IR360 surround convolution reverb internally use B-format reverb impulse responses.

\subsection{Scene Description Layer}

This layer mediates between the \emph{Authoring Layer} above and the \emph{Decoding Layer} below through an abstract and independent description about the spatial scene. This description can range from a simple static scene with one virtual sound sources up to complex dynamic audio scenes including multiple virtual spaces. These data could also be storable to recreate spatial scenes in a different context. Specific (lower-level) render instructions are communicated to the Encoding Layer beneath. Examples are ASDF \cite{geier2008ssr}, OpenAL \cite{openAL-spec1.1} or SpatDIF \cite{Peters:2008spatdif}.  	

\subsection{Authoring Layer}	

This layer contains all software tools for the end-user to create spatial audio content without the need to control underlying processes directly. Although these tools may differ remarkably through functionality and interface design from each other to serve the requirements for varicolored approaches to spatialization, the communication to the \emph{Scene Description Layer} must be standardized. 

Examples are symbolic authoring tools, generative algorithms, simulations of emergent behaviors (swarms or flock-of-birds); or more specifically, ICST ambimonitor/ambicontrol, and Holo-Edit.


\subsection{Winding up comments for chapter 3 TODO!!}

1) The idea we take from OSI is that for each layer (and for each level of communication between layers) there might be different ....
2) Some frendering techniques will operate acress more than one level. E.g. VBAP, DBAP, ViMiC and ambipanning incorporates encoding and decoding.
3) the flow of information/audio can go across several layers, the layers don't impose an absolute ordering and exclusive sequence of events.


%%%%%%%%%%%%%%%%%%%%%%%%%%%%%%%%%%%%%%%%%%%%%%%%%%%%%%%%%%%%%%%%%%%%%%%%%%%%%
%
%%%%%%%%%%%%%%%%%%%%%%%%%%%%%%%%%%%%%%%%%%%%%%%%%%%%%%%%%%%%%%%%%%%%%%%%%%%%%

\section{Stratified Tools}
%- structured according to layers   
In the following, several developments are discussed which are under development by the authors to establish and evaluate the proposed stratified concept.    
%In order to concretely experiment these ideas, the authors designed and used the tools described below :
%TODO: maybe the tools should be organized according to the layers %- hmmm... yes, but some of them are participating to different layers, e.g. ICST Ambisonics are in 5 and 3... Though, if we do so, maybe the description of the layers should contain the description of the tools used in our experimentations ????


\subsection{SpatDIF} 
The goal of the Spatial Sound Description Interchange Format (SpatDIF) is to develop a system-independent language for describing spatial audio \cite{Peters:2008spatdif} that can be applied around the \emph{Scene Description Layer} of Figure \ref{fig:layers}, to communicate between authoring tools down to the encoding/decoding layers. 
Formats that integrate spatial audio descriptors such as MPEG-4 Advanced Audio BIFS \cite{Vaananen:2004lr} or OpenAL did not fully succeed in the music or fine arts community because they are primary tailored to multimedia or gaming applications and don't necessarily consider the special requirements of spatial music, performances in concert venues, and site-specific media installations.
To account for these specific requirements, the SpatDIF development is consequently a collaborative effort that jointly involves researcher and artists.\\
\indent A database\footnote{\scriptsize{\url{http://redmine.spatdif.org/wiki/spatdif/SpatBASE}}} has been created to gather information about syntax and functionalities of common spatialization tools and to identify the lowest common denominator, the ``Auditory Spatial Gist'', for describing spatialized sound.\\ \indent Beside these essential \emph{Core Descriptors}, a number of extensions have been proposed to systematically account for enhanced features, e.g. the \emph{Directivity Extension} which deals with directivity information of a virtual sound source, the \emph{Acoustic Spaces Extension} that contains acoustical properties of virtual rooms, or the \emph{Ambisonics Extension} that handles ambisonics-only parameter. The \emph{Ambisonics Extension} is an example where SpatDIF mediates between the processing layers, starting from Layer 3 to Layer 6 (see Figure \ref{fig:layers}).\\ 
\indent Although SpatDIF does not imply a specific communication protocol or storing format, at present, OSC for streaming and SDIF \cite{schwarz2000eaa} as a storing solution are used for piloting. 
%                                                                                                            
\subsection{ICST Ambisonics}
The ICST Ambisonics Tools comprise a set of four externals for Max/MSP \cite{Schacher:2006ambi_max}.\\
\indent The two DSP externals ambiencode$\sim$ and ambidecode$\sim$ generate and decode Higher Order Ambisonics and are being part of the \emph{Encoding} and \emph{Decoding Layer} in Figure \ref{fig:layers}. They reach 3rd order with Furse-Malham Formulas in version 1.2, and 5th order using either the Furse-Malham or Normalized 3D formulas in version 2.0.\\ 
\indent The two externals ambimonitor and ambicontrol complete the set as control tools for the \emph{Authoring Layer}. Ambimonitor presents the user with a GUI displaying point sources in an abstract 2D or 3D space, various key commands, snapshot and file I/O capabilities and it generates coordinate information for the DSP objects. Ambicontrol provides a number of methods that control motion of points in the Ambimonitor's dataset. Automated motions such as rotation, random motion, optionally constrained in bounding volumes and user defined trajectories can be applied to single or grouped points. The import/export format for the trajectories and state snapshots is a simple XML text file, which will be replaced by a SpatDIF compliant formatting in a next release.\\
\indent A novel panning algorithm \cite{Neukom:2008ambipan} was derived from in-phase ambisonics decoding and was implemented as a Max/MSP external entitled ambipanning$\sim$. It encapsulates the \emph{Encoding} and \emph{Decoding Layer} by transcoding a set of mono sources in one process onto an ideally circular speaker setup with an arbitrary number of speakers. The algorithm works with a continuous order factor, which permits to apply individually varying directivity responses. This external understands the same command syntax and is therefore interchangeable with the two ICST Ambisonics Tools DSP objects. 
%
\subsection{Jamoma}

Jamoma\footnote{\url{http://www.jamoma.org}} is a open-source software\cite{Place:2006jamoma} to facilitate artistic and scientific work with Max/MSP. It consists of a DSP library and a modular framework for structuring patches.


Among others, the modular framework provides a set of popular spatialization algorithms, as there were VBAP \cite{Pulkki:2000vbap_max}, ambisonics \cite{Schacher:2006ambi_max}, ViMiC \cite{CMJ08-VIMIC}, and , DBAP \cite{dbapICMC09}. ViMiC and DBAP can be applied to non-conventional and non-standardized loudspeaker arrangements and therefore provides alternative solutions to the limitations outlined in section \ref{sec:digital audio workstations - DAW}. In particularly, DBAP has been proven to be useful in theaters and sound installation context.
  
\indent Furthermore, a number of spatial effects modules (Doppler, Airfilter, RollOff, Reverb), as well as multichannel recording \& playback modules, and other utilities to facilitate the work with multichannel audio (loudspeaker setup definition tools including sound-pressure and delay compensation for non-equidistant arrangements, multichannel auxiliaries, level-meters) are provided.
%that can be combined and substituted with each other to quickly compare, audition, or adapt each of them to a specific context. 
All these modules have been designed according to a common namespace model, that enables the user to rapidly combine or substituted modules in order to adapt to a specific requierements. In particular, the interoperability with Holo-Edit, ICST Ambimonitor and physical interfaces, such as Jazzmutant's Lemur %\footnote{\url{http://www.jazzmutant.com}} 
or TUIO-compliant \cite{kaltenbrunner2005tpt} multitouch devices through MMF\footnote{\url{http://code.google.com/p/mmf}} facilitates the individual control to Jamoma's rendering modules.


Current low-level spatial-systems development in Jamoma is focused on Jamoma Multicore, a coding layer for the creation of dynamic audio graph topologies. One application of Jamoma Multicore is the creation of ``multichannel signals" that can be passed between objects in Max/MSP or Pure Data. These multichannel signals are also dynamic, meaning that the number of channels can be changed instantaneously. % and the audio graph will adapt accordingly.  
The signals can also carry encoded audio signals such as the Ambisonics B-format of any order.
 
\subsection{Holo-Edit}
Initiated by L. Pottier \cite{PottierDAFX98}, Holo-Edit is part of the GMEM Holophon project and dedicated to multi-source and multi-speaker spatialization. This standalone application \cite{Pottier05} is written in Java / OpenGL and uses the timeline paradigm found in traditional DAWs for recording, editing, and playback of spatialization control messages. The data is manipulated in the form of trajectories or sequences of time-tagged points in a 3D space. These trajectories can be generated or modified by an expandable set of tools allowing specific spatial and temporal behaviors including symmetry, proportion, translation, acceleration, and local exaggeration. Different scene representation windows allow the user to view and modify data focusing on different aspects: \emph{Room} by a top view of the virtual space, \emph{Time Editor} which is related to the classic automation curve view in DAWs and \emph{Score} which represents the whole composition in a multi-track block-based view.


An important feature of Holo-Edit is the ability to represent sound waveforms beside % FIXME: Tim notes that this is not the correct use of the word beside, and I'm not really sure what you mean...
their trajectories. This allows the synchronization of sound cues to their associated movements.
The model of space/time representation has been left intentionally generic, thus it can control a variety of different spatialization methods and AVEs. %Even if Holo-Edit is combined with a auditory virtual environment such as \cite{CMJ08-VIMIC}, its generic nature allows to use it in a more abstract way, thinking more of movements in an arbitrary space ( may it be acoustic, symbolic, semiotic ...).[TODO: I changed something here, but I don't know what you really mean] [ FIXED BY CHARLES : I wanted to point that the movements composed are not made to be strongly linked with the scales of our physical world. But we can absolutly erase this line as it's not fundamental, was just a note ]
Holo-Edit uses OSC to communicate with various media programming environments presented in section \ref{sec:Media programming environments} and control various spatial renderers, including VBAP, Ambisonics, or DBAP. 
The main challenge in these scenarios is to adapt and format the data stream that fits the specific rendering algorithm syntax (e.g. coordinate system, dimensions, units).
As a step towards SpatDIF compliancy a Holo-edit communication interface was developed for Jamoma. It thus adds to the features of trajectory composition playback with the associated sound cues according to the edited score, by making provisions to record them from any real-time control inputs available in Jamoma. 
  
%%%%%%%%%%%%%%%%%%%%%%%%%%%%%%%%%%%%%%%%%%%%%%%%%%%%%%%%%%%%%%%%%%%%%%%%%%%%%
%
%%%%%%%%%%%%%%%%%%%%%%%%%%%%%%%%%%%%%%%%%%%%%%%%%%%%%%%%%%%%%%%%%%%%%%%%%%%%%

\section{Discussion \& Outlook}
This layered approach is an abstract and simplified description of spatial sound processes. Of course, there are other, and more complex and interacting processes embedded in a computer music environment, such as the access of memory and hard discs.   
TODO : Here we should maybe emphasize the fact that all these tools do work together and insert a screenshot of a patch demonstrating this (like the one I committed yesterday in UserLib/Holophon/Examples/LemurThingie) - and maybe a graph showing the connections... ???

The method used here tries to define standards through experimental implementation and usage :

This collaborative research has emerged from a workshop... we're planning to make more of them

The next step includes developing more use cases and a wider set of tools. SpatDIF and the modular approach to spatialization needs to grow through implementations in a variety of software, especially commercially used systems DAWs and plugins.



%%%% Do the following fit in here somewhere?

\begin{quote}Spatial orchestration eases this [...] constraint, [...] the composition may be reconfigured for each individual performance. This will require [...] the composer [to] conceive spatial attributes in a more abstract fashion that is then instantiated in potentially different ways into different performance environments. \cite{Lyon:2008spatial_orchestration} \end{quote}
By working symbolically during the composition/editing process, and separating the authoring from the rendering processes, a realization can become portable. A common set of descriptors and shared representation metaphors, such as timelines or scene graphs, builds the glue to make a disparate set of tools into a coherent workflow by enhancing interoperability between the building blocks.\

%%%%%%%%%%%%%%%%%%%%%%%%%%%%%%%%%%%%%%%%%%%%%%%%%%%%%%%%%%%%%%%%%%%%%%%%%%%%%
%
%%%%%%%%%%%%%%%%%%%%%%%%%%%%%%%%%%%%%%%%%%%%%%%%%%%%%%%%%%%%%%%%%%%%%%%%%%%%%

\section{Conclusion}
We proposed a multi-layer approach to mediate across essential components involved in sound spatialization. We showed that current spatialization approaches lack in terms of structure, interoperability and flexibility and argued that an improved structure will facilitate the artistic work. Finally, we showed examples of current developments which strive to employ this multi-layer approach........ \\                        
This work is partly funded by the Canadian Natural Sciences and Engineering Research Council (NSERC), the Canada Council for the Arts........... 
%%%%%%%%%%%%%%%%%%%%%%%%%%%%%%%%%%%%%%%%%%%%%%%%%%%%%%%%%%%%%%%%%%%%%%%%%%%%%
%
%%%%%%%%%%%%%%%%%%%%%%%%%%%%%%%%%%%%%%%%%%%%%%%%%%%%%%%%%%%%%%%%%%%%%%%%%%%%%

\section{Acknowledgment}
ICMC 2008 panel discussion \url{http://redmine.spatdif.org/wiki/spatdif/Belfast_2008}
\bibliographystyle{abbrv}  %\bibliographystyle{natbib}% ama, nar, alpha, plain, chicago,{plainnat}  abbrv, siam   
\small
\bibliography{smc09}
\end{document}   

%%%%%%%  Poubelle %%%%%%   
	
%\subsection{Frameworks for Spatialization}
%\textbf{Spatialisateur} (Spat) \cite{JotPhD}, in development at IRCAM and Espaces Nouveaux since 1991, is a library of spatialization algorithms for Max/MSP, including VBAP, first-order Ambisonics and stereo techniques (XY, MS, ORTF) for up to 8 loudspeakers. It can also reproduce 3D sound for headphones (binaural) or 2/4 loudspeakers (transaural). A room model is included to create artificial early reflections and reverb controlled by a perceptual-based user interface.\\
%\textbf{Ambisonics toolbox}  \cite{Schacher:2006ambi_max}

   

% \section{Style Stuff - delme}

% \subsection{Title and Authors}

%The title is 14pt Times, bold, caps, upper case, centered. Authors' names are centered. 
%The lead author's name is to be listed first (left-most), and the co-authors' names after. If the addresses for all 
%authors are the same, include the address only once, centered. If the authors have different addresses, 
%put the addresses, evenly spaced, under each authors' name.

% Reviews are double-blind. {\it \textbf{Author information should be removed from this page in initial submissions}}, 
% and only added later by the authors when sending camera-ready versions.


%\subsection{Figures, Tables and Captions}

%All artwork must be centered, neat, clean, and legible. All lines should
%be very dark for purposes of reproduction and art work should not be hand-drawn.
%The proceedings is not in color, and therefore all figures must make
%sense in black-and-white form.
%Figure and table numbers and captions always appear below the figure. Leave 1
%line space between the figure or table and the caption. Each figure or table
%is numbered consecutively. Captions should be Times 10pt.
%Place tables/figures in text as close to the reference as possible.
%References to figures and tables should be capitalised, for example:
%see Figure \ref{fig:example} and Table \ref{tab:example}.
%Figures and tables may extend across both columns to a maximum
%width of 7'' (17.78~cm).

%\begin{table}
%\begin{center}
%\begin{tabular}{|l|l|}
%\hline
%String value & Numeric value \\
%\hline
%hello SMC  & 1073 \\
%\hline
%\end{tabular}
%\end{center}
%\caption{Table captions should be placed below the table}
%\label{tab:example}
%\end{table}

%\begin{figure}
%\centerline{\framebox{
% To use when using pdflatex
%	\includegraphics[width=\columnwidth]{../ICMC2009-dbap/all_r_6_b_0_2}}}
	% To use when using latex, dvips and ps2pdf
% 	\includegraphics[width=\columnwidth]{figure.eps}}}
%\caption{Figure captions should be placed below the figure}
%\label{fig:example}
%\end{figure}




