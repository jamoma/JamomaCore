\chapter{Extending Objects}

\section{Subclassing Objects}

The classical methodology for extending functionality in object-oriented designs is to subclass an object.
In other words, we derive a new object that inherits all of the functionality of a parent class.
This is also possible in TTBlue, though you may wish to keep a few guidelines in mind when subclassing.

...



\section{Decorating Objects}

Because TTBlue is a dynamic and reflective API for creating objects, we can extend objects by means other than subclassing.
One specific way of doing this is by decorating a class.

For example, let us assume that we have a bandpass filter for processing audio.
Further, this object understands how to control its center frequency using an attribute specified in Hertz.
What are we to do if we want to communicate with the object using the Bark scale?

One option is to write a conversion routine and always call that routine to convert the value, then send a message the filter.
A second option is subclass the filter, creating another filter which has the required attribute specified using Barks.
A third option is to decorate the existing filter by adding a new attribute to it at runtime.

...




