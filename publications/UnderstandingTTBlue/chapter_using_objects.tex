\chapter{Using Objects}


\section{Object Life Cycle}

\subsection{Creating and Destroying}

To create an instance of a TTBlue object, use the TTObjectInstantiate().  
This will look up the class in TTBlue’s registry of objects and return a pointer to the new instance. 
 
An example below creates a stereo instance of an allpass filter, and two stereo audio signals.
They are stereo because the argument given when instantiating these types of objects define the initial number of audio channels as two.
You can safely ignore the fact that one of the variables is a TTAudioObjectPtr instead of a TTObjectPtr.  
TTAudioObject is simply a specialized version of TTObject.

\begin{small}\begin{verbatim}
	TTAudioObjectPtr  myObject      = NULL;
	TTObjectPtr       myAudioInput  = NULL;
	TTObjectPtr       myAudioOutput = NULL;

	TTObjectInstantiate(TT(“allpass”)), &myObject, 2);
	TTObjectInstantiate(TT(“audiosignal”)), &myAudioInput, 2);
	TTObjectInstantiate(TT(“ audiosignal”)), &myAudioOutput, 2);
\end{verbatim}\end{small}

When you are all done using the objects, you need to release them.  
You can do this with the TTObjectRelease() function.  
For example:

\begin{small}\begin{verbatim}
	TTObjectRelease(myObject);
	TTObjectRelease(myAudioInput);
	TTObjectRelease(myAudioOutput);
\end{verbatim}\end{small}


\subsection{Retaining and Reference Counts}

TTObject’s are reference counted.  This means that you can create references to existing objects, and the object will not be freed until all references have been released.  This is demonstrated in the following example:

\begin{small}\begin{verbatim}
	TTObjectPtr myObject = NULL;
	TTObjectPtr aReferenceToMyObject = NULL;

	TTObjectInstantiate(TT(“noise”)), &myObject, 2);
	// reference count is now 1
	aReferenceToMyObject = TTObjectReference(myObject);
	// reference count is now 2

	TTObjectRelease(myObject);
	// The object is not actually deleted, but the reference count is now 1
	TTObjectRelease(aReferenceToMyObject);
	// Now, because the reference count fell to zero, the object is deleted.
\end{verbatim}\end{small}



\section{Querying the Environment}

As alluded to in section 3.1, TTBlue maintains a registry of available classes that may be instantiated.  Some of these classes are implemented internally in the TTBlue library, and some are implemented externally as TTBlue Extensions.  The registry does not differentiate between these, but simply provides a list of everything that is loaded in the system.


\subsection{Getting All Class Names}

To obtain a list of all class in the TTBlue registry, you call the \lstinline!TTGetRegisteredClassNames()! function as in the following example:

%\begin{small}\begin{verbatim}
\begin{lstlisting}
	TTErr err;
	TTValue classNames;

	err = TTGetRegisteredClassNames(classNames);
	if(!err){
		// The classNames value now contains an array of TTSymbolPtrs, 
		// one for each class in the TTBlue registry.
		// For example, we can print them all to console:

		TTUInt16 numClassNames = classNames.getSize();

		for(TTUInt16 i=0; i<numClassNames; i++){
			TTSymbolPtr aClassName;

			classNames.get(i, aClassName);
			TTLogMessage(“class name: %s”, aClassName->getCString());
		}
	}
%\end{verbatim}\end{small}
\end{lstlisting}

\subsection{Searching For Classes Based on Tags}

In addition to retrieving all class names, it is also useful to be able to retrieve a limited number of class names based on criteria that you specify.  For example, you may wish to only list classes that generate their own audio.  Or perhaps only those class which implement some sort of lowpass filter.

In this case we call TTGetRegisteredClassNamesForTags(), and process the results in the same manner as we did for TTGetRegisteredClassNames().  This is demonstrated in the example below:

\begin{small}\begin{verbatim}
	TTErr err;
	TTValue classNames;
	TTValue searchTags;

	searchTags.clear();
	searchTags.append(TT(“audio”));
	searchTags.append(TT(“filter”));
	searchTags.append(TT(“lowpass”));

	err = TTGetRegisteredClassNamesForTags(classNames, searchTags);
	if(!err){
		// The classNames value now contains an array of TTSymbolPtrs, 
		// one for each class in the TTBlue registry.
		// For example, we can print them all to console:

		TTUInt16 numClassNames = classNames.getSize();

		for(TTUInt16 i=0; i<numClassNames; i++){
			TTSymbolPtr aClassName;

			classNames.get(i, aClassName);
			TTLogMessage(“class name: %s”, aClassName->getCString());
		}
	}
\end{verbatim}\end{small}

For a list of common tags and what they mean in TTBlue see Appendix A.  To get a list of all tags in use at any time, call TTGetRegisteredTags().


\subsubsection{Tag Searching and Instantiation in Action}

Here is an example that searches based on tags for a lowpass filter using a Butterworth algorithm, and then creates an instance of that class.

\begin{small}\begin{verbatim}
	// In this case there are multiple matches returned. 
	// Since more specific information was not provided, 
	// we just instantiate the first one.
	
	TTErr             err;
	TTValue           classNames;
	TTValue           searchTags;
	TTAudioObjectPtr  butterworthFilter = NULL;
	
	searchTags.clear();
	searchTags.append(TT(“audio”));
	searchTags.append(TT(“filter”));
	searchTags.append(TT(“lowpass”));
	searchTags.append(TT("butterworth"));
	
	err = TTGetRegisteredClassNamesForTags(classNames, searchTags);
	if(!err){
		// by passing classNames, the TTObjectInstantiate() function will take the first name in the list.
		err = TTObjectInstantiate(classNames, &butterworthFilter, 1);
	}
	// now do something with the filter.
\end{verbatim}\end{small}



%%%%%%%%%%%%%%%%%%%%%%%%%%%%%%%%%%%%%%%


\section{Sending Messages}

Having created an instance of an object, we must now do something with that object.
We do things with objects by sending them messages. 
A message defines an action to be performed. 

Given the Butterworth filter example in \ref{Tag Searching and Instantiation in Action}, we can now send the filter the 'clear' message to zero its sample history\footnote{The Butterworth filter is an IIR filter, meaning that it stores the results of previous calculations to perform future calculations.  This feedback can sometimes get out of control, and thus the necessity for a 'clear' message.}.

\begin{small}\begin{verbatim}
	butterworthFilter->sendMessage(TT("clear"));
\end{verbatim}\end{small}

Some messages, like the 'clear' message for the Butterworth filter, require no additional information to perform the requested action.
Other messages, however, do require additional information.  
This information can be provided using an optional argument to the sendMessage method.  Here are some samples:

\begin{small}\begin{verbatim}
	someObject->sendMessage(TT("foo"), 3.14);
	someObject->sendMessage(TT("draw"), TT("circle"));

	// create a TTValue that holds a list and pass the TTValue
	TTValue v(1.0, 2.0));
	someObject->sendMessage(TT("w"), v);
	
	// when you send a message, it can return a value as well.
	TTValue whatIsInThere;
	someObject->sendMessage("getContents", whatIsInThere);
	
	// Assuming that someObject understands the message, 
	// this results in whatIsInThere being set to contain something meaningful.	
\end{verbatim}\end{small}



%%%%%%%%%%%%%%%%%%%%%%%%%%%%%%%%%%%%%%%


\section{Setting and Getting Attributes}

Sending messages is great for performing actions, however these actions do not represent the state of the object.
The state of the object is represented as the object's 'attributes'.



\section{Querying Objects for Available Messages and Attributes}



\section{Processing Audio}

Not all objects can process audio, however TTBlue objects that have the tag ‘audio’ associated with them are able process audio.  These objects derive from TTAudioObject, a subclass of TTObject which implements the “process” method.



