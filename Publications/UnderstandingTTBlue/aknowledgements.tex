\chapter*{Aknowledgements}

The first version of TTBlue was originally written in the Autumn of 2003.  A large portion of the initial environment was created in the lobby of the Augustin Hotel in Bergen, Norway during series of workshops hosted by BEK\footnote{\url{http://bek.no/}}.  TTBlue was originally authored as the basis of Electrotap’s\footnote{\url{http://electrotap.com/}} Tap.Tools 2, a set of plug-ins for Cycling ’74’s Max/MSP\footnote{\url{http://cycling74.com/products/maxmsp/}} environment.  This in turn formed the basis of the Hipno\footnote{\url{http://cycling74.com/products/hipno/}} VST/AU/RTAS plug-ins released by Electrotap and Cycling ’74.  TTBlue was open-sourced in the Spring of 2005.

As an open source initiative, TTBlue has received valuable input from a large number of contributors in mostly informal ways.  
I can't hope to acknowledge everyone here.  Trond Lossius and Dave Watson are both active participants in the development of TTBlue and have added immeasurably to the environment, not just by the addition of code, but by their feedback and critique.  Joshua Kit Clayton also provided a valuable critique of TTBlue while we demonstrated Hipno at the Winter 2005 NAMM Show in Anaheim, California.

Many contributions to TTBlue are fed through the Jamoma\footnote{\url{http://jamoma.org/}} project.  In fact, many aspects of TTBlue were first developed by the Jamoma team and then ported to TTBlue, including the FunctionLib developed at a Jamoma workshop hosted by iMal\footnote{\url{http://imal.org/}}.  Through its support of Jamoma, GMEA\footnote{\url{http://gmea.net/}} has also helped indirectly in supporting TTBlue.

This document owes its creation in part Trond Lossius, who introduced me to \LaTeX (with which it is written), the Not So Short Introduction to \LaTeX (for which it is named), and also provided the basic structure for the document.  He is also largely responsible for the initial creation of the \footnote{\url{http://ttblue.org}} website, which is generously hosted by BEK.

